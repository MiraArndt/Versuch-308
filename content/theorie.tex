\section{Theorie}
\label{sec:Theorie}

Verweis auf Konstanten

Magnetfelder werden durch bewegte elektrische Ladungen hervorrufen. Der magnetische Fluss B wird in der Einheit Tesla 
angegeben und 


Da bewegte Ladungen magnetische Felder hervorrufen, lässt sich auch um stromdurchflossene Leiter ein 
magnetisches Feld messen, welches sich mithilfe von konzentrischen Kreisen, die senkrecht zum Leiter 
stehen, darstellen lässt.
Die magnetische Flussdichte B um eine beliebig geschlossene Leiterschleife lässt sich mithilfe des Biot-Savart-Gesetzes
\begin{equation}
    d\vec B = \frac{\mu_0 I}{4 \pi} \frac{d\vec s \times \vec r}{r^3}
    \label{eq:2} 
\end{equation}
\noindent
im Abstand r berechnen. Mithilfe dieser Formel lässt sich die magnetische Flussdichte auf einer Geraden, die durch den 
Mittelpunkt einer Leiterschleife/Ring verläuft ermitteln. Aus der Gleichung \ref{eq:1} ergibt sich 
\begin{equation}
    \vec B (x) = \frac{mu_0 I}{2} \frac{R^2}{(R^2+x^2)}^\frac{3}{2} \^{x},
\end{equation}
\noindent
wobei R den Radius des Ringes und x 
Liegt an Stelle eines Ringes eine Spule vor, so kann das Ergebnis mit der Windungszahl n multipliueirt werden.



Für den Fall, dass eine langegestreckte Spule mit l \gg D vorliegt, so verlaufen die Feldlinien 
innerhalb dieser parallel zueinander, was bedeutet, dass die magnetische Flussdichte homogen 
ist und sich dessen Betrag mit der Formel
\begin{equation}
    B = \mu_r \mu_0 \frac{n}{l} I
\end{equation}
\noindent 
berechnen lässt, wobei deutlich wird, dass die magnetische Flussdichte sowohl zu der Stromstärke I, 
als auch zu der Windungszahl n proportional und zu der Spulenlänge l umgekehrt proportional ist.

Für den Fall, dass eine Ringspule vorliegt, dessen Spulenradius r deutlich kleiner ist, als dessen Länge, lässt sich
der Betrag B mit der FOmel berechnen. Auch hier ist B iinen homogen, nur außen dieses mal null.

\subsection{Helmholtzspulen}

Um ein möglichst homogenes Magnetfeld zu erhalten, werden zwei identische Spulen mit dem Radius R und der Windungszahl N 
so aufgestellt, dass die Verbindungslinie der Spulenmittelpunkte senkrecht zu den Spulen selbst steht. Wird der Abstand 
der Spulen so eingestellt, dass dieser dem Radius R entspricht, so ergibt sich auf der Verbindungslinie der 
Spulenmittelpunkte ein homogenes Magnetfeld, für das sich die magnetische Flussdichte im Mittelpunkt der Spulen durch
die Überlagerung der einzelnen Felder ergibt. 
Weicht der Abstand der Spulen zueinander von dem Wert R ab, so ergibt sich der Wert für den magnetischen Fluss B im 
Mittelpunkt von zwei Spulen mit je einer Windung durch die Gleichung
\begin{equation}
    B(0) = B_1(x) + B_1(-x) = \frac{\mu_0 I R^2}{(R^2 + x^2)^(\frac{3}{2})},
\end{equation}
\noindent
wobei der Nullpunkt so gewählt ist, dass der Mittelpunkt der Helmholtzspulen in ihm liegt und der Wert x 
der Hälfte des Abstandes der Spulen zueinander entspricht.

\subsection{Hysteresekurve}
Im Gegensatz zu Dia- und Paramagneten besitzen Paramagneten bereits ohne ein von außen angelegtes Magnetfeld
ein permanentes magnetisches Moment. Innerhalb von sogenannten weißschen Bezirken verlaufen die magnetischen Momente
parallel, allerdings sind diese Bereiche statistisch über den gesamten Körper verteilt und heben sich gegenseitig auf,
sodass der Körper als ganzes kein magnetisches Feld besitzt. Wird nun ein äußeres Magnetfeld eingeschaltet, so ändert 
sich die Ausrichtung der einzelnen magnetischen Momente und die weißschen Bezirke vergrößern sich. Das äußere Magnetfeld 
lässt sich so weit erhöhen, bis alle magnetischen Momente die gleiche Ausrichtung aufweisen. Als Beispiel für ein
ferromagnetisches Material lässt sich Eisen nennen.
\noindent
Bedingt dadurch, dass bei Ferromagneten die relative Permeabilität $\mu_r \ll 0$ ist, verliert Gleichung (VERWEIS AUF GLEICHUNG)
ihre Gültigkeit. Um den Zusammenhang zwischen magnetischer Erregung und magnetischem Fluss dennoch darstellen zu können, wird
eine sogenannte Hysteresekurve erstellt. Diese gibt auf der x-Achse den Wert der magnetischen Erregung und auf der y-Achse
den Wert der magnetischen Flussdichte an. 
Wird ein ferromagnetisches Material zum ersten Mal durch ein äußeres Magnetfeld beeinflusst, vergrößern sich, wie zuvor
bereits erwähnt, die weißschen Bezirke, bis alle magnetischen Momente gleich ausgerichetet sind. An diesem Punkt erreicht 
der magnetische Fluss seinen maximalen Wert. Man spricht von der sogenannten Sättigungsmagnetisierung $U_s$. Wird
das äußere Magnetfeld nun wieder verkleinert, so ändert ein Teil der magnetischen Momente innerhalb der Ferromagneten 
seine Ausrichtung. In Folge dessen nimmt der Wert für die magnetische Flussdichte ebenfalls ab. Ist das äußere Magnetfeld
vollständig abgeschaltet, kann jedoch beobachtet werden, dass das ferromagnetische Material selbst trotzdem noch ein magnetisches
Feld aufweist. Dies liegt daran, dass immernoch der überwiegende Teil der magnetischen Momente in ein und dieselbe Richtung 
ausgerichetet ist. Dieser Wert wird auch als Remanenz bezeichnet. Es ist also zu beachten, dass die magnetische Flussdichte 
des Ferromagneten nicht nur von dem äußeren magnetischen Feld abhängt, sondern auch von dem Verlauf seiner vorhergegangenen 
Magnetisierung. Wenn nun ein äußeres Magnetfeld angelegt wird, welches dem ursprünglichen entgegengerichtet ist, wird der Zustand
der statistischen Verteilung der weißschen Bezirke wiederhergestellt und das Magnetfeld des Körpers selbst verschwindet wiederum.
Dieser Punkt auf der x-Achse wird als Koerzitivkraft bezeichnet. Wird das äußere Magnetfeld nun so weit vergrößtert, bis sich alle 
magnetischen Momente in dem Körper wieder gleich ausgerichetet haben, weist der magnetische Fluss des Ferromagneten bis auf das Vorzeichen 
denselben Wert  


Symmetrisch zum Ursprung
Vorgeschichte







\cite{sample}

\section{Theorie}
\label{sec:Theorie}

Verweis auf Konstanten

Magnetfelder werden durch bewegte elektrische Ladungen hervorrufen. Der magnetische Fluss B wird in der Einheit Tesla 
angegeben und 


Da bewegte Ladungen magnetische Felder hervorrufen, lässt sich auch um stromdurchflossene Leiter ein 
magnetisches Feld messen, welches sich mithilfe von konzentrischen Kreisen, die senkrecht zum Leiter 
stehen, darstellen lässt.
Die magnetische Flussdichte B um eine beliebig geschlossene Leiterschleife lässt sich mithilfe des Biot-Savart-Gesetzes
\begin{equation}
    d\vec B = \frac{\mu_0 I}{4 \pi} \frac{d\vec s \times \vec r}{r^3}
    \label{eq:1} 
\end{equation}
\noindent
im Abstand r berechnen. Mithilfe dieser Formel lässt sich die magnetische Flussdichte auf einer Geraden, die durch den 
Mittelpunkt einer Leiterschleife/Ring verläuft ermitteln. Aus der Gleichung \ref{eq:1} ergibt sich 
\begin{equation}
    \vec B (x) = \frac{mu_0 I}{2} \frac{R^2}{(R^2+x^2)}^\frac{3}{2} \^{x},
\end{equation}
\noindent
wobei R den Radius des Ringes und x 
Liegt an Stelle eines Ringes eine Spule vor, so kann das Ergebnis mit der Windungszahl n multipliueirt werden.



Für den Fall, dass eine langegestreckte Spule mit l \gg D vorliegt, so verlaufen die Feldlinien 
innerhalb dieser parallel zueinander, was bedeutet, dass die magnetische Flussdichte homogen 
ist und sich dessen Betrag mit der Formel
\begin{equation}
    B = \mu_r \mu_0 \frac{n}{l} I
\end{equation}
\noindent 
berechnen lässt, wobei deutlich wird, dass die magnetische Flussdichte sowohl zu der Stromstärke I, 
als auch zu der Windungszahl n proportional und zu der Spulenlänge l umgekehrt proportional ist.

Für den Fall, dass eine Ringspule vorliegt, dessen Spulenradius r deutlich kleiner ist, als dessen Länge, lässt sich
der Betrag B mit der FOmel berechnen. Auch hier ist B iinen homogen, nur außen dieses mal null.

\subsection{Helmholtzspulen}

Um ein möglichst homogenes Magnetfeld zu erhalten, werden zwei identische Spulen mit dem Radius R und der Windungszahl N 
so aufgestellt, dass die Verbindungslinie der Spulenmittelpunkte senkrecht zu den Spulen selbst steht. Wird der Abstand 
der Spulen so eingestellt, dass dieser dem Radius R entspricht, so ergibt sich auf der Verbindungslinie der 
Spulenmittelpunkte ein homogenes Magnetfeld, für das sich die magnetische Flussdichte im Mittelpunkt der Spulen durch
die Überlagerung der einzelnen Felder ergibt. 
Weicht der Abstand der Spulen zueinander von dem Wert R ab, so ergibt sich der Wert für den magnetischen Fluss B im 
Mittelpunkt von zwei Spulen mit je einer Windung durch die Gleichung
\begin{equation}
    B(0) = B_1(x) + B_1(-x) = \frac{\mu_0 I R^2}{(R^2 + x^2)^(\frac{3}{2})},
\end{equation}
\noindent
wobei der Nullpunkt so gewählt ist, dass der Mittelpunkt der Helmholtzspulen in ihm liegt und der Wert x 
der Hälfte des Abstandes der Spulen zueinander entspricht.









\cite{sample}

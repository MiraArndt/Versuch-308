\section{Diskussion}
Bei der langen Spule konnten die Werte in der Mitte
der Spule nicht ausreichend gut gemessen werden, da
die Hall-Sonde zu kurz war. Somit ist nicht klar erkennbar,
ob das Magnetfeld im Innneren der Spule konstant bleibt.
Auf jeden Fall ist ab einer gewissen Tiefe, in
der die Randeffekte nachlassen, zu erkennen, dass
die Steigung der magnetischen Flussdichte stark abnimmt und
die B-Feld-Kurve somit abflacht, doch der errechnete Theoriewert wird
beim gemessenen Bereich nicht erreicht. Außerdem ist noch anzumerken,
dass wahrscheinlich ein Fehler bei der Messung vorlag, denn
an Abbildung \ref{fig:a}  ist zu erkennen, dass beim
Nullpunkt ein ungewöhnlich großer Sprung auftritt. 
Abgesehen davon ist das B-Feld nahezu symmetrisch,
was auch den Erwartungen entspricht.

Dies ist ebenfalls bei der kurzen Spule der Fall.
Bei der Durchführung ist jedoch bereits aufgefallen,
dass die Windungen der Spule etwas locker waren und
diese somit nicht gleichmäßig verteilt waren.

Die Ergebnisse der Messung zu der Helmholtzspule entsprechen
im Allgemeinen den Erwartungen. Ungenauigkeiten
können hier dadurch zustande gekommen sein, dass
die Haltevorrichtung, die die Hall-Sonde auf der
Symmetrieachse halten sollte etwas locker war und 
ein paar Millimeter Spielraum ließ. Außerdem musste
die Hall-Sonde senkrecht zum Magnetfeld per Hand
ausgerichtet werden, was die Ungenauigkeit der Messung erhöht.

Die Neukurve der Spule mit Eisenkern 
zeigt nicht exakt dem Verlauf, der in der Theorie vorhergesagt wurde.
Dies lässt sich jedoch dadurch erklären, dass die verwendete
Spule schon vor dem Versuch oft magnetisiert wurde und
dementsprechend eine Restmagnetisierung von
\begin{equation}
    B=107,4\,\si{\milli\tesla}\nonumber
\end{equation}
\noindent vorlag.
Abgesehen davon entspricht die Hysteresekurve vollkommen den
Erwartungen.